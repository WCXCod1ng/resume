% !TEX TS-program = xelatex
% !TEX encoding = UTF-8 Unicode
% !Mode:: "TeX:UTF-8"

\documentclass{resume}
\usepackage{zh_CN-Adobefonts_external} % Simplified Chinese Support using external fonts (./fonts/zh_CN-Adobe/)
% \usepackage{NotoSansSC_external}
% \usepackage{NotoSerifCJKsc_external}
% \usepackage{zh_CN-Adobefonts_internal} % Simplified Chinese Support using system fonts
\usepackage{linespacing_fix} % disable extra space before next section
\usepackage{cite}

\begin{document}
\pagenumbering{gobble} % suppress displaying page number

\name{王晨曦}

\basicInfo{
  \email{3331056621@qq.com} \textperiodcentered\ 
  \phone{(+86) 15189468735} \textperiodcentered\ 
  https://github.com/WCXCod1ng}
 
\section{\faGraduationCap\  教育背景}
\datedsubsection{\textbf{南京大学}, 南京, 江苏}{2024 -- 至今}
\textit{在读硕士研究生}\ 计算机科学与技术, 预计 2027 年 6 月毕业
\datedsubsection{\textbf{南京审计大学}, 南京, 江苏}{2020 -- 2024}
\textit{学士}\ 计算机科学与技术

\section{\faCogs\ IT 技能}
% increase linespacing [parsep=0.5ex]
\begin{itemize}[parsep=0.5ex]
  \item 掌握C++11,并熟悉C++14/17/20常见的新特性;了解Golang、Python等其他编程语言
  \item 熟悉HTTP、TCP、UDP等常见网络协议,了解Linux环境下的Socket编程、高性能网络开发模型
  \item 熟悉Redis的基本用法,理解其常见数据类型、分布式锁等;掌握MySQL的增删改查操作,了解事务、索引等
  \item 熟悉STL中vector、map、set等常见容器的源码;了解常见的设计模式
  % \item 会使用C++相关的编译构建工具、调试工具等
\end{itemize}

\section{\faUsers\ 实习/项目经历}
\datedsubsection{\textbf{基于C++20协程的高性能Web框架}}{2025年9月 -- 2025年12月}
\role{个人项目}{C++}
以muduo库作为蓝本,使用Epoll的ET模式取代原有的LT模式;在此基础上实现了包含基于Radix Tree的HTTP路由机制、AOP中间件机制和数据库连接池的后端Web框架的开发;使用C++20引入的协程机制提升Web框架业务处理的并发能力。
\begin{itemize}
  \item 将muduo库中的Epoll的LT模式重构为ET模式,减少系统调用次数,提升高并发下的吞吐量。
  \item 设计并实现了基于Radix Tree的HTTP路由分发机制,在路径查找时间复杂度上优于传统的正则匹配或哈希表查找;实现了基于责任链模式的中间件机制,支持全局与路由组级别的切面处理(日志、鉴权等);集成MySQL连接池。
  \item 除了引入业务线程池避免阻塞IO操作,还引入了C++20的协程机制,并实现了协程调度器以提升业务处理的并发能力。
  \item 使用对象池解决业务处理过程中频繁的堆分配开销。
\end{itemize}

\datedsubsection{\textbf{南京荣耀软件研究所}}{2025年7月 -- 2025年9月}
\role{实习}{二进制逆向}
\begin{onehalfspacing}
作为主力参与二进制逆向Agent平台的研发与实现。该平台基于混合分析和多Agent的机制,能够自动化地对给定的安卓native库文件进行动静态混合分析,并利用多Agent架构对这些信息进行推理,实现对目标二进制函数的函数名恢复任务。
\begin{itemize}
  \item 搭建安卓native库的执行环境:通过集成Frida,实现对目标二进制函数执行信息的捕获、对二进制函数间调用关系的捕获;通过结合Agent,实现对指定目标二进制函数的动态执行
  \item 实现基于Ghidra的静态分析流程的构建,包括提取汇编、伪C代码、签名、静态调用图;设计调用图增强算法利用动态调用信息完善静态调用图
  \item 通过在调用图上按照逆拓扑序遍历,以及动态更新调用图,赋予Agent感知目标函数的调用关系的能力,以提升恢复效果。
\end{itemize}
\end{onehalfspacing}

% \datedsubsection{\textbf{\LaTeX\ 简历模板}}{2015 年5月 -- 至今}
% \role{\LaTeX, Python}{个人项目}
% \begin{onehalfspacing}
% 优雅的 \LaTeX\ 简历模板, https://github.com/billryan/resume
% \begin{itemize}
%   \item 容易定制和扩展
%   \item 完善的 Unicode 字体支持,使用 \XeLaTeX\ 编译
%   \item 支持 FontAwesome 4.5.0
% \end{itemize}
% \end{onehalfspacing}

% Reference Test
%\datedsubsection{\textbf{Paper Title\cite{zaharia2012resilient}}}{May. 2015}
%An xxx optimized for xxx\cite{verma2015large}
%\begin{itemize}
%  \item main contribution
%\end{itemize}

\section{\faHeartO\ 竞赛经历}
\datedline{全国二等奖,蓝桥杯算法竞赛}{2022年12月}
\datedline{江苏省二等奖,全国大学生数学建模竞赛}{2022年12月}

% \section{\faInfo\ 其他}
% % increase linespacing [parsep=0.5ex]
% \begin{itemize}[parsep=0.5ex]
%   \item 语言: 英语 - 六级
% \end{itemize}

%% Reference
%\newpage
%\bibliographystyle{IEEETran}
%\bibliography{mycite}
\end{document}
